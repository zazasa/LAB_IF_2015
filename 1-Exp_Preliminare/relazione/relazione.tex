\documentclass[a4paper,10pt]{article}
%\usepackage[english]{babel}   % se si scrive in inglese
\usepackage[italian]{babel}       % se si scrive in italiano
\usepackage[T1]{fontenc}       % serve per la codifica delle lettere accentate (per evitare di scrivere \`e) 
\usepackage[utf8]{inputenc}   % serve per la codifica delle lettere accentate (per evitare di scrivere \`e)

\usepackage{graphicx}  % per includere le figure
\usepackage{rotating}   
\usepackage{lscape}
\usepackage{amsmath}
\usepackage{mathrsfs}
\usepackage[abs]{overpic}
\usepackage{color}
\usepackage{layout}
\usepackage[textwidth=16cm,textheight=24cm]{geometry} 


%% Definizione di nuovi commandi %%%
\newcommand{\latex}{\LaTeX}
\newcommand{\CP}{{\rm CP}}
\newcommand{\proton}{\ensuremath{p}}
\newcommand{\antiproton}{\ensuremath{\bar{p}}}
\newcommand{\acp}{\ensuremath{{\cal A}_{\CP}}}
\newcommand{\Lbppi}{\ensuremath{\Lambda_{b}^{0} \rightarrow p\pi^{-}}}
\newcommand{\aLbppi}{\ensuremath{\bar{\Lambda}_{b}^{0} \rightarrow \bar{p}\pi^{+}}}
\newcommand{\acplbppi}{\ensuremath{\acp(\Lbppi)}}
\newcommand{\lambdazero}{\ensuremath{\Lambda}}
\newcommand{\alambdazero}{\ensuremath{\bar{\Lambda}}}
\newcommand{\lambdazeroppi}{\ensuremath{\lambdazero \to \proton\pi^-}}
\newcommand{\alambdazeroppi}{\ensuremath{\bar{\lambdazero} \to \antiproton \pi^+}}

\newcommand{\pt}{\ensuremath{p_{\rm{T}}}}
\newcommand{\lxy}{\ensuremath{L_{\rm{T}}}}
\newcommand{\pdf}{\ensuremath{\wp}}

\newcommand{\mum}{\mbox{$\mu$m}}				%	um
\newcommand{\mus}{\mbox{$\mu$s}}
\newcommand{\tev}{\ensuremath{\mathrm{Te\kern -0.1em V}}}
\newcommand{\gev}{\ensuremath{\mathrm{Ge\kern -0.1em V}}}	%	GeV
\newcommand{\mev}{\ensuremath{\mathrm{Me\kern -0.1em V}}}	%	MeV
\newcommand{\kev}{\ensuremath{\mathrm{ke\kern -0.1em V}}}	%	keV
\newcommand{\massgev}{\mbox{\gev/$c^2$}}			%	GeV/c^2
\newcommand{\massmev}{\mbox{\mev/$c^2$}}			%	MeV/c^2
\newcommand{\pgev}{\mbox{\gev/$c$}}				%	GeV/c
\newcommand{\pmev}{\mbox{\mev/$c$}}				%	MeV/c

\newcommand{\note}[1]{\textcolor{red}{#1}}

%%%%%%%%%%%%%%%%%%%%%%%%%%
%%%%%% BEGIN DOCUMENT %%%%
%%%%%%%%%%%%%%%%%%%%%%%%%%
\begin{document}
\begin{flushright}             
\today
\end{flushright} 

\begin{center}
\Large{\bf Condizioni di lavoro di un contatore a scintillazione. }

\vspace*{1cm}                                 
\large{Michael Denuccio, Fabio Spagliardi, Giacomi Vitali, Salvatore Zaza }\\ 
\vspace*{0.5cm}       
\vspace*{1.cm}
\end{center}

%{ \abstract  
%}


\section{Introduzione}
\label{sec:intro} 
Lo scopo dell'esperienza è quello di acquisire pratica e mettere a punto un sistema di tre rilevatori a scintillatore plastico. Infine valutare un opportuno punto di lavoro per poi misurare l'efficienza di uno dei tre.

%%%%%%%%%%%%%%%%%%%%%%%%%%%%%%%%%%%%%%%%%%%%%%%%%%%%%%%%%%%%%%%%%%%%%%%%%%%%%%%%
%%%%%%%%%%%%%%%%%%%%%%%%%%%%%%%%%%%%%%%%%%%%%%%%%%%%%%%%%%%%%%%%%%%%%%%%%%%%%%%%
%%%%%%%%%%%%%%%%%%%%%%%%%%%%%%%%%%%%%%%%%%%%%%%%%%%%%%%%%%%%%%%%%%%%%%%%%%%%%%%%
\section{Raggi cosmici}
\note{Un po' di teoria sui raggi cosmici e una giustificazione sul numero di muoni attesi}

\section{Apparato Sperimentale}
In figura \ref{fig:none} è schematizzano l'apparato sperimentale. E' composto da tre scintillatori collegati tramite una guida d'onda ad un fotomoltiplicatore (PMT). Faremo riferimento ai 3 tramite il numero associato al loro PMT (come in figura).
\\I PMT sono alimentati tramite un modulo dell'alta tensione (HV). 
\\Per l'osservazione e l'acquisizione dei segnali dai PMT abbiamo utilizzato i seguenti strumenti elettronici:
\begin{itemize}
\item Multimetro digitale;
\item Oscilloscopio a 2 canali;
\item Modulo FAN-IN FAN-OUT per lo smistamento dei segnali;
\item Discriminatore NIM a 8 canali;
\item Diversi moduli per il delay dei segnali;
\item Un modulo NIM per le coincidenze;
\item Un contatore digitale NIM con timer;
\end{itemize}


\label{sec:apparato} 
\subsection{Scintillatore Plastico}
Uno scintillatore plastico è composto da una lastra di materiale organico capace di emettere fotoni quando viene attraversato da una particella ionizzante. 

\subsection{PhotoMultiplier Tube (PMT)}

\subsection{Elettronica di Front-End}


%%%%%%%%%%%%%%%%%%%%%%%%%%%%%%%%%%%%%%%%%%%%%%%%%%%%%%%%%%%%%%%%%%%%%%%%%%%%%%%%
%%%%%%%%%%%%%%%%%%%%%%%%%%%%%%%%%%%%%%%%%%%%%%%%%%%%%%%%%%%%%%%%%%%%%%%%%%%%%%%%
%%%%%%%%%%%%%%%%%%%%%%%%%%%%%%%%%%%%%%%%%%%%%%%%%%%%%%%%%%%%%%%%%%%%%%%%%%%%%%%%


\section{Ricerca del punto di lavoro}
\label{sec:puntodilavoro} 
\subsection{Calibrazione del discriminatore e del contatore}
\note{descrivere la misura effettuata sul trigger del discriminatore e il controllo del timer del contatore}

\subsection{Osservazioni preliminari}
Inizialmente abbiamo alimentate il PMT4 con una tensione di 1.700 V, osservando un assorbimento in corrente di 0,693 mA. Collegando il segnale all'oscilloscopio ed impostando una soglia di trigger di 30 mV abbiamo visualizzato impulsi della durata dai 25 ai 31 ns, di ampiezza molto variabile dai 30 ai 750 mV, ad una frequenza dell'ordine dei kHz.
\\
Quindi abbiamo collegato il PMT ad un contatore tramite un discriminatore, impostato con una soglia a 50mV e una larghezza del segnale di uscita di 40ns. In questo modo abbiamo preso le misure mostrate nel grafico \ref{fig:none}, che mostra una netta differenza nei conteggi in caso di luce accesa o spenta, segnalandoci la possibile presenza di imperfezioni nella copertura esterna.
\\
Nelle prossime misure abbiamo cercato di limitare questo effetto ricoprendo l'apparato con un panno nero.
\\
\note{Per le misure dei conteggi abbiamo scelto di adottare un tempo di acquisizione di 100s per ...}
\\
\note{Misura del guadagno del PMT...}
%%%%%%%%%%%%%%%%%%%%%%%%%%%%%%%%%%%%%%%%%%%%%%%%%%%%%%%%%%%%%%%%%%%%%%%%%%%%%%%%
%%%%%%%%%%%%%%%%%%%%%%%%%%%%%%%%%%%%%%%%%%%%%%%%%%%%%%%%%%%%%%%%%%%%%%%%%%%%%%%%
%%%%%%%%%%%%%%%%%%%%%%%%%%%%%%%%%%%%%%%%%%%%%%%%%%%%%%%%%%%%%%%%%%%%%%%%%%%%%%%%
\subsection{Curva in funzione della tensione di alimentazione}
Abbiamo alimentato tutti e 3 PMT con una tensione variabile da 1600 a 1900 mV. Il segnale dei 3 PMT lo abbiamo collegato ad un contatore passando attraverso un discriminatore per il conteggio degli impulsi.\\
Osservando il segnale sull'oscilloscopio abbiamo deciso di impostare una soglia di 50 mV, ed una larghezza del segnale di uscita del discriminatore di 40 ns. 
Il grafico \ref{fig:none} mostra il numero dei conteggi in funzione della tensione di alimentazione. Si può osservare come i PMT 4 e 6 mostrano un andamento molto simile mentre il PMT 5 sembra essere molto più sensibile.
\\
Da queste misure risulta difficile individuare un plateau, ma ci sono comunque tornate utili in futuro per la scelta di un punto di lavoro che presenti una minima variazione del tasso dei conteggi in funzione della tensione.
\\
Il numero dei conteggi e molto variabile e non ci permette di stabilire una relazione col numero di raggi cosmici attesi. D'altra parte ci aspettiamo che un singolo scintillatore presenti una quantità di conteggi dovuti al rumore di fondo molto maggiore del numero di conteggi dovuti al passaggio di particelle ionizzanti, ed è per questo che possiamo giustificare conteggi anche molto elevati.

\subsection{Calibrazione delle coincidenze}
Utilizzando i dati in tabella [mettere tabella foglio excel 1], abbiamo calcolato il rate di frequenze casuali attese tra il PMT 4 e 6. In questo modo abbiamo stabilito che mantenendoci al di sotto di una soglia di alimentazione di 1900 V il numero di coincidenze casuali si mantiene comunque basso per non inficiare in maniera significativa sul conteggio di coincidenze reali che ci aspettiamo.

\subsubsection{Ritardo di cavo}

\subsubsection{Misura delle coincidenze casuali}


\section{Misura dell'efficienza}
\note{definizione dell'efficienza}
\subsection{Risultati}
\label{sec:efficienza} 

%%%%%%%%%%%%%%%%%%%%%%%%%%%%%%%%%%%%%%%%%%%%%%%%%%%%%%%%%%%%%%%%%%%%%%%%%%%%%%%%
%%%%%%%%%%%%%%%%%%%%%%%%%%%%%%%%%%%%%%%%%%%%%%%%%%%%%%%%%%%%%%%%%%%%%%%%%%%%%%%%
%%%%%%%%%%%%%%%%%%%%%%%%%%%%%%%%%%%%%%%%%%%%%%%%%%%%%%%%%%%%%%%%%%%%%%%%%%%%%%%%

\section{Conclusioni}
%%%%%%%%%%%%%%%%%%%%%%%%%%%%%%%%%%%%%%%%%%%%%%%%%%%%%%%%%%%%%%%%%%%%%%%%%%%%%%%%
%%%%%%%%%%%%%%%%%%%%%%%%%%%%%%%%%%%%%%%%%%%%%%%%%%%%%%%%%%%%%%%%%%%%%%%%%%%%%%%%
%%%%%%%%%%%%%%%%%%%%%%%%%%%%%%%%%%%%%%%%%%%%%%%%%%%%%%%%%%%%%%%%%%%%%%%%%%%%%%%%

%\appendix


\end{document}
