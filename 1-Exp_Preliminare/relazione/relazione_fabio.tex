\documentclass[a4paper,10pt]{article}
%\usepackage[english]{babel}   % se si scrive in inglese
\usepackage[italian]{babel}       % se si scrive in italiano
\usepackage[T1]{fontenc}       % serve per la codifica delle lettere accentate (per evitare di scrivere \`e) 
\usepackage[utf8]{inputenc}   % serve per la codifica delle lettere accentate (per evitare di scrivere \`e)

\usepackage{graphicx}  % per includere le figure
\usepackage{rotating}   
\usepackage{lscape}
\usepackage{amsmath}
\usepackage{mathrsfs}
\usepackage[abs]{overpic}
\usepackage{color}
\usepackage{layout}
\usepackage[textwidth=16cm,textheight=24cm]{geometry} 


%% Definizione di nuovi commandi %%%
\newcommand{\latex}{\LaTeX}
\newcommand{\CP}{{\rm CP}}
\newcommand{\proton}{\ensuremath{p}}
\newcommand{\antiproton}{\ensuremath{\bar{p}}}
\newcommand{\acp}{\ensuremath{{\cal A}_{\CP}}}
\newcommand{\Lbppi}{\ensuremath{\Lambda_{b}^{0} \rightarrow p\pi^{-}}}
\newcommand{\aLbppi}{\ensuremath{\bar{\Lambda}_{b}^{0} \rightarrow \bar{p}\pi^{+}}}
\newcommand{\acplbppi}{\ensuremath{\acp(\Lbppi)}}
\newcommand{\lambdazero}{\ensuremath{\Lambda}}
\newcommand{\alambdazero}{\ensuremath{\bar{\Lambda}}}
\newcommand{\lambdazeroppi}{\ensuremath{\lambdazero \to \proton\pi^-}}
\newcommand{\alambdazeroppi}{\ensuremath{\bar{\lambdazero} \to \antiproton \pi^+}}

\newcommand{\pt}{\ensuremath{p_{\rm{T}}}}
\newcommand{\lxy}{\ensuremath{L_{\rm{T}}}}
\newcommand{\pdf}{\ensuremath{\wp}}

\newcommand{\mum}{\mbox{$\mu$m}}				%	um
\newcommand{\mus}{\mbox{$\mu$s}}
\newcommand{\tev}{\ensuremath{\mathrm{Te\kern -0.1em V}}}
\newcommand{\gev}{\ensuremath{\mathrm{Ge\kern -0.1em V}}}	%	GeV
\newcommand{\mev}{\ensuremath{\mathrm{Me\kern -0.1em V}}}	%	MeV
\newcommand{\kev}{\ensuremath{\mathrm{ke\kern -0.1em V}}}	%	keV
\newcommand{\massgev}{\mbox{\gev/$c^2$}}			%	GeV/c^2
\newcommand{\massmev}{\mbox{\mev/$c^2$}}			%	MeV/c^2
\newcommand{\pgev}{\mbox{\gev/$c$}}				%	GeV/c
\newcommand{\pmev}{\mbox{\mev/$c$}}				%	MeV/c

\newcommand{\note}[1]{\textcolor{red}{#1}}

%%%%%%%%%%%%%%%%%%%%%%%%%%
%%%%%% BEGIN DOCUMENT %%%%
%%%%%%%%%%%%%%%%%%%%%%%%%%
\begin{document}
\begin{flushright}             
\today
\end{flushright} 

\begin{center}
\Large{\bf Condizioni di lavoro di un contatore a scintillazione. }

\vspace*{1cm}                                 
\large{Michael De Nuccio, Fabio Spagliardi, Giacomi Vitali, Salvatore Zaza }\\ 
\vspace*{0.5cm}       
\vspace*{1.cm}
\end{center}

%{ \abstract  
%}


\section{Introduzione}
\label{sec:intro} 
Lo scopo dell'esperienza è quello di acquisire pratica e mettere a punto un sistema di tre rilevatori a scintillatore plastico. Infine valutare un opportuno punto di lavoro per poi misurare l'efficienza di uno dei tre.

%%%%%%%%%%%%%%%%%%%%%%%%%%%%%%%%%%%%%%%%%%%%%%%%%%%%%%%%%%%%%%%%%%%%%%%%%%%%%%%%
%%%%%%%%%%%%%%%%%%%%%%%%%%%%%%%%%%%%%%%%%%%%%%%%%%%%%%%%%%%%%%%%%%%%%%%%%%%%%%%%
%%%%%%%%%%%%%%%%%%%%%%%%%%%%%%%%%%%%%%%%%%%%%%%%%%%%%%%%%%%%%%%%%%%%%%%%%%%%%%%%
\section{Raggi cosmici}
\note{Un po' di teoria sui raggi cosmici e una giustificazione sul numero di muoni attesi}

\section{Apparato Sperimentale}
In figura \ref{fig:none} è schematizzano l'apparato sperimentale. E' composto da tre scintillatori collegati tramite una guida d'onda ad un fotomoltiplicatore (PMT). Faremo riferimento ai 3 tramite il numero associato al loro PMT (come in figura).
\\I PMT sono alimentati tramite un modulo dell'alta tensione (HV). 
\\Per l'osservazione e l'acquisizione dei segnali dai PMT abbiamo utilizzato i seguenti strumenti elettronici:
\begin{itemize}
\item multimetro digitale;
\item oscilloscopio a 2 canali;
\item modulo FAN-IN FAN-OUT per lo smistamento dei segnali;
\item discriminatore NIM a 8 canali;
\item diversi moduli per il delay dei segnali;
\item un modulo NIM per le coincidenze;
\item un contatore digitale NIM con timer;
\item cavi di diversa lunghezza con impedenza standardizzata di 50 $\Omega$.
\end{itemize}


\label{sec:apparato} 

\subsection{Scintillatore Plastico}
Uno scintillatore plastico è composto da una lastra di materiale plastico in cui è dissolta una sostanza organica capace di emettere fotoni quando viene attraversato da una particella ionizzante. Si tratta di rilevatori veloci che si prestano bene ad essere utilizzati per realizzare contatori: infatti i tempi tipici sono dell'ordine del nanosecondo. La luce emessa dalla diseccitazione degli elettroni eccitati dal passaggio della particella ionizzante si propaga all'interno del rivelatore attraverso il meccanismo della riflessione totale. Il fotone emesso non viene riassorbito grazie al principio di Franck-Condon. La luce si riflette fino all'estremità dello scintillatore dove continua a propagarsi in una guida di luce, la quale permette di raccordare la superficie del rivelatore con il fotomoltiplicatore. \\
Nel nostro caso i tre scintillatori sono ricoperti da scotch nero. Ciò permette di isolare dalla luce ambientale e, allo stesso tempo, si riesce a lasciare uno strato d'aria tra la superficie esterna del rilevatore e lo scotch: ciò è ottimale per avere riflessione totale alle superficie.

\subsection{PhotoMultiplier Tube (PMT)}
Ognuno dei tre scintillatori è collegato ad un fotomoltiplicatore. Non si dispone di un datasheet per il modello utilizzato.\\
Un fotomoltiplicatore, in generale, è formato da una sequenza di elettrodi (detti dinodi, la cui geometria può essere molto variabile). Agli estremi di questa sequenza sono presenti un anodo e un catodo. Il fotone che esce dallo scintillatore si trova ad interagire con il catodo tramite effetto fotoelettrico (effetto dominante considerate le energie a cui si lavora, ovvero qualche KeV). Grazie al campo elettrico introdotto tramite la differenza di potenziale tra catodo e anodo i primi, e relativamente pochi, elettroni prodotti vengono accelerati e, mediante ionizzazione per urto con gli elettrodi interposti, si innesca un fenomeno a cascata che porta ad aumentare il numero di elettroni che vengono via via accelerati. Per essere più quantitativi si può considerare il numero di elettroni che vengono estratti ad ogni elettrodo: $\delta=2\div 5$; il guadagno totale in elettroni sarà quindi $G=\delta ^N$ dove $N$ è il numero totale di dinodi. Solitamente il catodo viene messo a massa e l'anodo viene tenuto ad una tensione elevata (1000-2000 $V$); tramite un partitore di tensione i dinodi vengono mantenuti a tensioni intermedie opportune. Alla fine di questo processo si osserva un segnale elettrico macroscopico.\\
Per stimare l'ampiezza di questo segnale si può ricorrere ad un calcolo veloce: tipicamente si trova che il numero di fotoni prodotti è $2\times 10^{4}$ per centimento di spessore del rilevatore. Facendo un calcolo indicativo si ottiene il numero di fotoni che effettivamente giungono al fotomoltiplicatore è circa $1400$, numero ottenuto considerando l'efficacia di raccolta della luce da parte della guida, il rapporto tra le interfacce di scintillatore e PMT e l'attenuazione esponenzialmente decrescente con la distanza, durante la propagazione. Si definisce il parametro $\eta$, detto "efficienza quantica", come la frazione di fotoni che riescono a fare un effetto fotoelettrico efficace. Di conseguenza, moltiplicando per $\eta$, si ottengono 350 fotoelettroni. Se si considera come guadagno tipico di un PMT $G=\delta ^N=3^{12}=5 \times 10^5$ si ottengono all'anodo $1,8 \times 10^8$ elettroni, ovvero una carica di 29 pC. Considerato che questa carica è spalmata in un intervallo temporale dell'ordine di 10 ns si ottiene un'intensità di corrente $I=3$ mA che equivale, con una resistenza standardizzata all'anodo di $50\ \Omega$, ad un segnale dell'ordine di 150 mV. Questo tipo di segnale è facilmente visualizzabile e misurabile tramite un oscilloscopio.



\subsection{Elettronica di Front-End}
L'elettronica di Front-End è composta dai moduli, digitali e analogici, indicati sopra. Il segnale in uscita dal fotomoltiplicatore è stato tipicamente mandato al modulo per il delay regolabile dei segnali. L'uscita del modulo per il delay veniva inviata al modulo FAN-IN FAN-OUT per lo smistamento dei segnali, facendo attenzione alle terminazioni. Questa configurazione iniziale permette di ritardare a piacimento il segnale e a formarne quattro copie che possono essere mandatie ai vari moduli o analizzanti con l'oscilloscopio. Un segnale veniva poi inviato al discriminatore il quale, una volta impostata la soglia, converte il segnale in ingresso in un segnale digitale secondo la logica NIM. Il modulo che abbiamo utilizzato permette sia di regolare il livello del discriminatore, sia la durata dell'impulso (piatto) associato. La scelta di questi parametri verrà specificata nel seguito. A questo punto si è usato il contatore digitale NIM con timer regolabile per contare gli impulsi digitali in uscita dal discriminatore. Tutto quanto detto verrà fatto per ogni singolo fotomoltiplicatore utilizzato. 







\section{Ricerca del punto di lavoro}
\label{sec:puntodilavoro} 
\subsection{Calibrazione del discriminatore e del contatore}
\note{descrivere la misura effettuata sul trigger del discriminatore e il controllo del timer del contatore}

\subsection{Osservazioni preliminari}
Inizialmente abbiamo alimentato il PMT4 con una tensione di 1.700 V, osservando un assorbimento in corrente di 0,693 mA. Collegando il segnale all'oscilloscopio ed impostando una soglia di trigger di 30 mV abbiamo visualizzato impulsi della durata dai 25 ai 31 ns, di ampiezza molto variabile dai 30 ai 750 mV, ad una frequenza dell'ordine dei kHz.
\\
Quindi abbiamo collegato il PMT ad un contatore tramite un discriminatore, impostato con una soglia a 50mV e una larghezza del segnale di uscita di 40ns. In questo modo abbiamo preso le misure mostrate nel grafico \ref{fig:none}, che mostra una netta differenza nei conteggi in caso di luce accesa o spenta, segnalandoci la possibile presenza di imperfezioni nella copertura esterna.
\\
Nelle prossime misure abbiamo cercato di limitare questo effetto ricoprendo l'apparato con un panno nero.
\\
\note{Per le misure dei conteggi abbiamo scelto di adottare un tempo di acquisizione di 100s per ...}
%%%%%%%%%%%%%%%%%%%%%%%%%%%%%%%%%%%%%%%%%%%%%%%%%%%%%%%%%%%%%%%%%%%%%%%%%%%%%%%%
%%%%%%%%%%%%%%%%%%%%%%%%%%%%%%%%%%%%%%%%%%%%%%%%%%%%%%%%%%%%%%%%%%%%%%%%%%%%%%%%
%%%%%%%%%%%%%%%%%%%%%%%%%%%%%%%%%%%%%%%%%%%%%%%%%%%%%%%%%%%%%%%%%%%%%%%%%%%%%%%%
\subsection{Curva in funzione della tensione di alimentazione}
Abbiamo alimentato tutti e 3 PMT con una tensione variabile da 1600 a 1900 mV. Il segnale dei 3 PMT lo abbiamo collegato ad un contatore passando attraverso un discriminatore per il conteggio degli impulsi.\\
Osservando il segnale sull'oscilloscopio abbiamo deciso di impostare una soglia di 50 mV, ed una larghezza del segnale di uscita del discriminatore di 40 ns. 
Il grafico \ref{fig:none} mostra il numero dei conteggi in funzione della tensione di alimentazione. Si può osservare come i PMT 4 e 6 mostrano un andamento molto simile mentre il PMT 5 sembra essere molto più sensibile.
\\
Da queste misure risulta difficile individuare un plateau, ma ci sono comunque tornate utili in futuro per la scelta di un punto di lavoro che presenti una minima variazione del tasso dei conteggi in funzione della tensione.
\\
Il numero dei conteggi e molto variabile e non ci permette di stabilire una relazione col numero di raggi cosmici attesi. D'altra parte ci aspettiamo che un singolo scintillatore presenti una quantità di conteggi dovuti al rumore di fondo molto maggiore del numero di conteggi dovuti al passaggio di particelle ionizzanti, ed è per questo che possiamo giustificare conteggi anche molto elevati.

\subsection{Calibrazione delle coincidenze}
Utilizzando i dati in tabella [mettere tabella foglio excel 1], abbiamo calcolato il rate di frequenze casuali attese tra il PMT 4 e 6. In questo modo abbiamo stabilito che mantenendoci al di sotto di una soglia di alimentazione di 1900 V il numero di coincidenze casuali si mantiene comunque basso per non inficiare in maniera significativa sul conteggio di coincidenze reali che ci aspettiamo.

\subsubsection{Ritardo di cavo}

\subsubsection{Misura delle coincidenze casuali}


\section{Misura dell'efficienza}
\note{definizione dell'efficienza}
\subsection{Risultati}
\label{sec:efficienza} 

%%%%%%%%%%%%%%%%%%%%%%%%%%%%%%%%%%%%%%%%%%%%%%%%%%%%%%%%%%%%%%%%%%%%%%%%%%%%%%%%
%%%%%%%%%%%%%%%%%%%%%%%%%%%%%%%%%%%%%%%%%%%%%%%%%%%%%%%%%%%%%%%%%%%%%%%%%%%%%%%%
%%%%%%%%%%%%%%%%%%%%%%%%%%%%%%%%%%%%%%%%%%%%%%%%%%%%%%%%%%%%%%%%%%%%%%%%%%%%%%%%

\section{Conclusioni}
%%%%%%%%%%%%%%%%%%%%%%%%%%%%%%%%%%%%%%%%%%%%%%%%%%%%%%%%%%%%%%%%%%%%%%%%%%%%%%%%
%%%%%%%%%%%%%%%%%%%%%%%%%%%%%%%%%%%%%%%%%%%%%%%%%%%%%%%%%%%%%%%%%%%%%%%%%%%%%%%%
%%%%%%%%%%%%%%%%%%%%%%%%%%%%%%%%%%%%%%%%%%%%%%%%%%%%%%%%%%%%%%%%%%%%%%%%%%%%%%%%









%\appendix














\end{document}
