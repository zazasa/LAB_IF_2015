\documentclass[a4paper,9pt]{article}
\usepackage[italian]{babel}
%\usepackage[textwidth=15cm,textheight=20cm]{geometry}
\usepackage[textwidth=16cm,textheight=24cm]{geometry}

%\pagestyle{empty}

\usepackage{array}
\usepackage{graphicx}
\usepackage{lscape}
\usepackage{mathrsfs}
\usepackage[abs]{overpic}
\usepackage{amsmath}
\usepackage{amssymb}
\usepackage{lscape}
\usepackage{rotating}
\usepackage{hyperref}



% \oddsidemargin .4cm
% \topmargin 0cm
% \headsep .5cm
% \textheight 23.5cm
% \textwidth 15.9cm
% \topskip .6cm
% \setlength{\unitlength}{1mm}

\renewcommand{\baselinestretch}{1.2}
\renewcommand{\arraystretch}{1.9}
\newcommand{\sep}{\hspace{0.1cm}}
\newcommand{\back}{\hspace*{-1.5cm}}



\begin{document}

\begin{center}
{\Large Esercitazione per il corso di Laboratorio di Interazioni Fondamentali  \\
%\vspace{0.5cm}
Anno Accademico 2015/2016\\
\vspace{0.2cm}
Laurea Magistrale in Fisica dell'Universit\`a di Pisa}\\
\end{center}

\vspace{1cm}
\noindent
\textit{Nota: Questa esperienza di analisi dati \`e stata ideata dal Dr Sergio Giudici per il Corso di Laboratorio 
di Interazioni Fondamentali del Prof. Marco Sozzi del Corso di Laurea Magistrale in Fisica dell'Universit\`a di Pisa 
nell'anno accademico 2013/14, che si trova in
\href{http://www.df.unipi.it/~giudici/analisi_dati.html}{http://www.df.unipi.it/$\sim$giudici/analisi\_dati.html}.
Mentre nella versiona originale il fit al cerchio, per stimare i parametri $R,x_0, y_0$, veniva effetuato analiticamente con 
il metodo dei minimi quadrati, in questo caso si vuole utilizzare il software MINUIT nel framework ROOT per  
minimizzare l'estimatore.}

\vspace{1cm}
Un fascio di particelle di energia 120 GeV/$c^2$ contiene mesoni $K^+$ e $\pi^+$ rispettivamente di massa 0.493~GeV/$c^2$ e 0.139~GeV/$c^2$. 
Tramite un rivelatore ad effetto Cerenkov piazzato lungo il fascio \`e  possibile discriminare tra le due particelle.
I fotoni ottici sono prodotti quando la particella carica del fascio attraversa un recipiente di lunghezza 1000~cm riempito di gas opportuno. 
I fotoni sono riflessi da uno specchio parabolico di lunghezza focale pari a 1000 cm posto al fondo del recipiente e vengono riflessi sul piano focale. 
Il piano focale \`e segmentato in 50 x 50 pixel (1cm x 1cm ) otticamente attivi che registrano l'arrivo dei fotoni ottici.
A valle del contatore Cerenkov, a 9000 cm da esso, \`e posto uno scintillatore che registra il passaggio dei muoni emessi nel corso dei decadimenti
 $K,\pi \to \mu \nu_{\mu}$. Lo scintillatore è una corona circolare di raggio interno 10 cm e raggio esterno 100 cm.
\vspace{0.5cm}
%%%%%%%%%%%%%%%
\begin{figure}[h] 
\begin{center} 
%\small
%\footnotesize
\scriptsize 
\begin{overpic}[width=7cm]{./cerenkov_schema.pdf}
\put(-25,82){$K,\pi$}
\put(-40,70){120~GeV/$c^2$}
\put(17,5){z=0 cm}
\put(93,5){z=1000 cm}
\put(169,5){z=10000 cm}
\put(-10,140){Piano focale}
\put(16,52){Radiatore Cerenkov}
\put(70,127){Specchio}
\put(70,117){parabolico}
\put(70,107){f=1000 cm}
\put(155,148){Scintillatore}
\put(110,90){$\mu$}
\put(122,83.5){$\theta$}
%\put(197,35){$\theta_{\rm max}$}
%\put(88,150){$E$}
\end{overpic}   
\hspace{1cm}
\begin{overpic}[width=7cm]{./cerenkov_grid.pdf}
\put(28,30){\rotatebox{90}{$i_y=1,2,3,.................,50.$}}
\put(55,15){$i_x=1,2,3,.................,50.$}
%\put(88,35){$\theta_0$}
%\put(56,35){$\theta_{\rm min}$}
%\put(197,35){$\theta_{\rm max}$}
%\put(88,150){$E$}
\end{overpic}   
\label{fig:schema}
\caption{Schema.}
\end{center} 
\end{figure} 
%%%%%%%%%%%%%%
%%%%%%%
I pixel otticamente attivi posti sul piano focale sono identificati secondo un sistema cartesiano.
%%%%%%%%
%%%%%%%
%% METTERE FIGURA %%%
%%%%%%%%%
%%%%%%%
%\landscape{$i=1,2,3,.................50$}
Il fascio passa nel centro del quadrato 50 x 50. Ogni pixel  \`e identificato dal numero intero $$i_x \cdot 1000 + i_y = s.$$ 
Fate l'esercizio di trovare le coordinate intere $i_x$ e $i_y$ dato l'intero $s$. Allo scopo \`e utile ricordare l'operazione 
aritmetica ``modulo'' (a \texttt{mod} b). Nel file \texttt{cerenkov.dat.gz} (da unzippare con il comando linux ``gunzip'', 
per ogni evento sono date le seguenti informazioni:
\begin{itemize}
\item numero di tracce cariche viste dallo scintillatore,
\item numero di pixel colpiti,
\item coordinate dei pixel colpiti nella forma $i_x \cdot 1000 + i_y$.
\end{itemize}
Per esempio, un evento con 1 muone e 10 pixel colpiti ha la forma\\
1\\ 
10\\ 
2013\\
7043\\
7044\\ 
9045\\ 
10003\\ 
11003\\ 
12002\\ 
12047\\ 
15049\\ 
23050.\\

\noindent
Si dimostri che nel corso di un evento
\begin{enumerate}
\item[1.] i fotoni ottici si dispongono sul piano focale lungo un cerchio.
\item[2.] e si ricavi un'espressione per il raggio del cerchio in funzione della massa della particella e dell'indice di rifrazione del gas contenuto nel contatore.
\end{enumerate}
Seguendo l'algoritmo di fit ad un cerchio descritto nella nota allegata, si ricostruisca per ciascun evento il raggio dell'anello Cerenkov e si trovi:
\begin{enumerate}
\item[3.] la frazione di mesoni K contenuti nel fascio (stimando l'errore),
\item[4.] l'indice di rifrazione del gas contenuto nel contatore (stimando l'errore),
\item[5.] si dimostri che il muone emesso dal mesoni $\pi$ non pu\`o mai essere visto dallo scintillatore,
\item[6.] si definisca un intervallo per il raggio dell'anello $r_1 < r < r_2$ che definisce i mesoni $K$ e si stimi la probabilit\`a che un mesone $K$, 
  non sia identificato e la probabilit\`a che un mesone $\pi$ sia confuso con un mesone $K$. Quali accorgimenti si potrebbero adottare per ridurre 
i casi di identificazione sbagliata ?
\end{enumerate}


\thispagestyle{empty} %% toglie il numero di pagina

\end{document}





